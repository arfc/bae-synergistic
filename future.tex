\section{Future Nuclear Projections}

The future of nuclear energy in \gls{EU} nations is organized
in the table by the World Nuclear Association \cite{world_nuclear_association_nuclear_2017}.
It is assumed in the simulations that all the planned constructions 
are completed on their expected date without delay
or failure. Also, the newly constructed nuclear power plants are assumed to have a lifetime of 60 years.

\Cref{tab:eu_deployment} lists the reactors that are currently  planned or under construction.

 
\begin{table}[h]
	\centering
	\caption {Power Reactors under construction and planned \cite{world_nuclear_association_nuclear_2017}}
	\label{tab:eu_deployment}
	\scalebox{0.70}{
	\begin{tabular}{|c|c|c|c|c|}
		\hline
		Exp. Operational & Country & Reactor & Type & Gross MWe\\
		\hline
		2018 & Slovakia  & Mochovce 3 & PWR & 440\\
		2018 & Slovakia & Mochovce 4 & PWR & 440 \\
		2018 & France & Flamanville 3 & PWR & 1600 \\
		2018 & Finland & Olkilouto 3 & PWR & 1720 \\		
		2019 & Romania & Cernavoda 3 & PHWR & 720 \\
		2020 & Romania & Cernavoda 4 & PHWR & 720 \\
		2024 & Finland & Hanhikivi & VVER1200 & 1200 \\
		2024 & Hungary & Paks 5 & VVER1200 & 1200 \\
		2025 & Hungary & Paks 6 & VVER1200 & 1200 \\
		2025 & Bulgaria & Kozloduy 7 & AP1000? & 950 \\
		2026 & UK & Hinkley Point C1 & EPR & 1670 \\
		2027 & UK & Hinkley Point C2 & EPR & 1670 \\
		2029 & Poland & Choczewo & N/A & 3000 \\
		2035 & Poland & N/A & N/A & 3000 \\
		2035 & Czech Rep & Dukovany 5 & N/A & 1200 \\
		2035 & Czech Rep & Temelin 3 & AP1000 & 1200 \\
		2040 & Czech Rep & Temelin 4 & AP1000 & 1200 \\
		\hline
	\end{tabular}
	}
\end{table}
\FloatBarrier

For each \gls{EU} nation, the growth trajectory is categorized from
``Aggressive Growth'' to ``Aggressive Shutdown''. Aggressive growth is
characterized by a rigorous expansion of nuclear power while 
Aggressive Shutdown is characterized as a transition to rapidly
de-nuclearize the nation's electric grid. A nation's growth trajectory is
categorized into five degrees depending on G, the growth trajectory metric.
  
 \[
 G = \left\{\begin{array}{lr}
 \text{Aggressive Growth}, & \text{for } G \geq 2\\
 \text{Modest Growth}, & \text{for } 1.2 \leq G < 2\\
 \text{Maintanence}, & \text{for } 0.8 \leq G < 1.2 \\
 \text{Modest Reduction}, & \text{for } 0.5 \leq G< 0.8\\
 \text{Aggressive Reduction}, & \text{for } G \leq 0.5
 \end{array}\right\} = \frac{C_{2040}}{C_{2017}}\\\\
 \]
 \[
  G = \text{Growth Trajectory} [-] 
 \]
 \[
 C_{i} = \text{Nuclear Capacity in Year i  } [MWe]
 \]
 
The growth trajectory and specific plan of each nation in the \gls{EU} 
is listed in Table \ref{tab:eu_growth}.

\begin{table}[h]
	\centering
		\begin{tabularx}{\textwidth}{smb}
			\hline 
			
			Nation & Growth Trajectory & Specific Plan \\
			\hline \hline
			UK & Aggressive Growth & {\small 13 units (17,900 MWe) by 2030.}\\
			\hline
			Poland & Aggressive Growth & {\small Additional 6,000 MWe by 2035.}\\
			\hline
			Hungary & Aggressive Growth & {\small Additional 2,400 MWe (VVER-1200) by 2025.} \\ 
			\hline
			Finland & Modest Growth & {\small Additional EPR in 2018, VVER in 2024.}\\
			\hline
			Bulgaria & Modest Growth & {\small Additional AP1000 (1,000 MWe) construction in 2035. }\\
			\hline
			Romania & Modest Growth & {\small Additional 1,440 MWe by 2020.} \\
			\hline
			Czech Rep. & Modest Growth & {\small Additional 2,400 MWe (AP1000s) by 2035.}\\
			\hline
			France & Maintenance & {\small Shutdown nuclear plants if they reach end of lifetime. No new construction.}\\
			\hline
			Spain & Modest Reduction & {\small No plans to expand or early shutdown.} \\
			\hline
			Italy & Modest Reduction & {\small No plans to expand or early shutdown.} \\
			\hline
			Belgium & Aggressive Reduction & All shut down 2025.\\
			\hline
			Sweden & Aggressive Reduction & All shut down 2050.\\
			\hline
			Germany & Aggressive Reduction & All shut down by 2022.\\
			\hline
			
		\end{tabularx}

	\caption {Future Nuclear Programs of \gls{EU} Nations \cite{world_nuclear_association_nuclear_2017}}
  \label{tab:eu_growth}
\end{table}
\FloatBarrier
