\section{Conclusion}

This work demonstrates that France can transition into
a fully \gls{SFR} fleet with installed capacity of 66,000 \gls{MWe} without
building additional \glspl{LWR}
if France receives \gls{UNF} from other \gls{EU} nations.
Supporting the \gls{SFR} fleet requires an average 
reprocessing capacity of 63.23 \gls{MTHM} per month,
and an average fabrication capacity of 45.32 \gls{MTHM} per month.

The sensitivity study explored the effect of increased \gls{SFR} breeding
ratio and existing \gls{LWR} lifetime extension. Increasing the breeding
ratio reduced the amount of \gls{LWR} \gls{UNF} required to transition
up to $9.7\%$ and decreased the total reprocessing demand up to $3.4\%$.
Increasing the lifetime of existing \glspl{LWR} 
increased reprocessing demand, since more \gls{LWR} \gls{UNF} were
reprocessed for plutonium prior to \gls{ASTRID} deployment.

Since most \gls{EU} nations do not have an operating \gls{UNF}
repository or a management plan, they have a strong incentive
to send their \gls{UNF} to France. In particular, the nations
planning aggressive nuclear reduction will be able phase out nuclear
without constructing a permanent repository. France has an
incentive to take this fuel, since recycling used fuel from
other nations will allow France to meet their MOX demand
without new construction of \glspl{LWR}.

Table \ref{tab:which_send} lists \gls{EU} nations and their \gls{UNF} inventory
in 2050. We analyzed a strategy in which 
the nations reducing their nuclear fleet send their \gls{UNF} to France.
The sum of \gls{UNF} from Italy, Slovenia, Belgium, Spain and Germany
provides enough \gls{UNF} for the simulated transition ($\approx 53,000$ MTHM). 
These nations are shown in bold in table \ref{tab:which_send}.
Sweden is not considered because of its concrete waste management plan.

If France receives \gls{LWR} \gls{UNF} from all \gls{EU} nations,
except Sweden and Finland,
it will have a surplus of $30,648$ MTHM of \gls{LWR} \gls{UNF}. This
inventory can be leveraged to increase nuclear power capacity as
the transition takes place. However, pragmatic limitations such
as new reactor construction, reprocessing throughput, and
political concerns remain.

\begin{table}[h]
    \centering
    \caption {\gls{EU} nations and their respective \gls{UNF} inventory.} 
                \begin{tabularx}{\textwidth}{llr}
                    \hline 
                    \textbf{Nation} & \textbf{Growth Trajectory} & \small{\textbf{UNF in 2050 [MTHM] }}\\
                    \hline
                    Poland & Aggressive Growth & 1,807\\
                    Hungary & Aggressive Growth & 3,119 \\ 
                    UK & Aggressive Growth & 13,268\\
                    Slovakia & Modest Growth & 2,746\\
                    Bulgaria & Modest Growth & 3,237 \\
                    Czech Rep. & Modest Growth & 4,413\\
                    Finland & Modest Growth &  5,713\\
                    Netherlands & Modest Reduction & 539\\
                    \textbf{Italy} & \textbf{Modest Reduction} & \textbf{583}\\
                    \textbf{Slovenia} & \textbf{Modest Reduction} & \textbf{765}\\
                    Lithuania & Modest Reduction & 2,644 \\
                    \textbf{Belgium} & \textbf{Aggressive Reduction} & \textbf{6,644}\\
                    \textbf{Spain} & \textbf{Modest Reduction} &  \textbf{9,771} \\
                    \textbf{France} & \textbf{Modest Reduction} & \textbf{12,494} \\
                    Sweden & Aggressive Reduction & 16,035\\
                    \textbf{Germany} & \textbf{Aggressive Reduction} & \textbf{23,868}\\
                    \hline
                \end{tabularx}
    
    \label{tab:which_send}

\end{table}

On the other hand, in these simulations, some complex political and economic factors were not incorporated and various assumptions were present in this scenario. For
example, Germany's current policy is to not reprocess its \gls{LWR} fuel
\cite{topfer_germanys_2011}, and this policy would create a shortage
in the supply of \gls{LWR} \gls{UNF} for \gls{ASTRID} \gls{MOX} production.
Continuation of that German policy would not, however, be incompatible
with a change in \gls{EU} policy that frees \gls{EU} countries from
creating their high level waste repositories, since France could still
agree to take in Germany's \gls{UNF} for direct disposal. The analysis
method described herein could readily be adapted to account for such possibilities. 
The collaborative option explored here may hold value for the \gls{EU} nuclear community,
and may enable France to advance more rapidly into a closed fuel cycle. 
\FloatBarrier


\section{Future Work}

This analysis relies on many simplifying assumptions concerning 
reactor physics and depletion. Future work implementing higher 
fidelity fuel burnup and depletion models may more faithfully resolve 
plutonium vector changes in  \gls{ASTRID} and \gls{MOX} \gls{PWR} cores. 
Additionally, incorporating true isotopic vectors for initial and end-of-life 
core compositions will improve accuracy for many metrics of interest.
