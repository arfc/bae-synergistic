\section{Discussion}
This work demonstrates that France can transition into
a fully \gls{SFR} fleet with installed capacity of 66,000 MWe without
building additional \glspl{LWR}
if France receives \gls{UNF} from other \gls{EU} nations.
Supporting the \gls{SFR} fleet would require a reprocessing capacity of 250 MTHM per month,
and a fabrication capacity of 200 MTHM per month.

Since most \gls{EU} nations do not have an operating \gls{UNF}
repository or a management plan, they have a strong incentive
to send all their \gls{UNF} to France. The nations
with aggressive nuclear reduction will be able phase out nuclear
without constructing a permanent repository. France has an
incentive to take this fuel, since recycling of used fuel from
other nations will allow France to meet their MOX demand
without new construction of \glspl{LWR}.

\Cref{tab:which_send} proposes a strategy where nations
that are planning a nuclear phaseout send their \gls{UNF}
to France. The sum of \gls{UNF} from the four countries
provide enough \gls{UNF} for the simulated transition.
The four nations would then be able to decommission
their reactors without building a permanent repository. 

\begin{table}[h]
    \centering
                \begin{tabularx}{\textwidth}{lbb}
                       \hline 
                    
                    \textbf{Nation} & \textbf{Growth Trajectory} & \small{\textbf{UNF in 2050 [MTHM] }}\\
                    \hline
                    UK & Aggressive Growth & 53,188\\
                    \hline
                    Poland & Aggressive Growth & 6,714\\
                    \hline
                    Hungary & Aggressive Growth & 4,768 \\ 
                    \hline
                    Finland & Modest Growth &  7,528\\
                    \hline
                    Slovakia & Modest Growth & 3,446\\
                    \hline
                    Bulgaria & Modest Growth & 3,930 \\
                    \hline
                    Czech Rep. & Modest Growth & 7,583\\
                    \hline
                    Slovenia & Modest Reduction & 765\\
                    \hline
                    Netherlands & Modest Reduction & 539\\
                    \hline
                    Lithuania & Modest Reduction & 2,644 \\
                    \hline
                    \textbf{France} & \textbf{Modest Reduction} & \textbf{12,943} \\
                    \hline 
                    \textbf{Spain} & \textbf{Modest Reduction} &  \textbf{9,771} \\
                    \hline
                    \textbf{Italy} & \textbf{Modest Reduction} & \textbf{583}\\
                    \hline
                    \textbf{Belgium} & \textbf{Aggressive Reduction} & \textbf{6,644}\\
                    \hline
                    Sweden & Aggressive Reduction & 16,035\\
                    \hline
                    \textbf{Germany} & \textbf{Aggressive Reduction} & \textbf{23,868}\\
                    \hline
                \end{tabularx}
    \caption {\gls{EU} nations and their respective \gls{UNF} inventory. The bolded countries'
              \gls{UNF} inventory adds up to the required \gls{UNF} amount for French \gls{SFR} transition. }
    \label{tab:which_send}
\end{table}

Though complex political and economic factors are overlooked,
 and various assumptions present for this scenario,
this option may hold value for the \gls{EU} as a nuclear community,
and for France to advance into a closed fuel cycle.
