\section{Discussion}
This work demonstrated that, given infinite
reprocessing and \gls{MOX} fabrication capacities,
France, by receiving \gls{UNF} from other \gls{EU} nations,
 can transition into, for unchanging nuclear electricity demand,
a fully \gls{SFR} fleet
with installed capacity of 66,000 MWe by 2076.
\gls{MOX} from reprocessed \gls{UNF} meets the initial fuel demand,
which later on is supplied by \gls{MOX} created from recycled \gls{MOX}.

Since most \gls{EU} nations do not have an operating \gls{UNF}
repository or a management plan, they have a strong incentive
to send all their \gls{UNF} to France. The nations
with aggressive nuclear reduction can phase out nuclear
without constructing a High Level Waste repository. France has an
incentive to take this fuel, since reuse of used fuel from
other nations will allow France to meet their MOX demand
without new construction of \glspl{LWR}.

Though complex political and economic factors are not
addressed, and various assumptions present for this scenario,
this option may hold value for the \gls{EU} as a nuclear community,
and for France to advance into a closed fuel cycle.
