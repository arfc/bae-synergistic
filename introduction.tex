
\section{Introduction}
The stated long term plan for nuclear deployment in France targets a technology 
transition to \glspl{SFR}\cite{cne2_reports_2015}. However, the current inventory of French \gls{UNF} 
is insufficient to fuel that transition without building new \glspl{LWR}.

If instead, France accepted 
\gls{UNF} from other \gls{EU} nations and used it to produce \gls{MOX} for new \glspl{SFR},
the \gls{MOX} created will fuel a French transition to an \gls{SFR} fleet
and allow France to avoid building additional \glspl{LWR}.


We used the \Cyclus nuclear fuel cycle simulator \cite{huff_fundamental_2016} to simulate
 \gls{EU} spent nuclear material inventory accumulation and to model the 
 proposed French 
 technology transition from \glspl{LWR} to
 \glspl{SFR}. \Cyclus is an agent-based extensible
framework for modeling the flow of material through future nuclear fuel cycles.
We calculated the used fuel
inventory in \gls{EU} member states and propose a potential collaborative
strategy of used fuel management.


Past research focuses solely on France and typically assumes that additional \glspl{LWR},
namely \glspl{EPR}, supply the \gls{UNF} required to produce \gls{MOX} 
\cite{carre_overview_2009, martin_symbiotic_2017, freynet_multiobjective_2016, merino_rodriguez_analysis_2014}.
The strategies in these works estimate full \gls{SFR} transition in 2100.
Other recent work in the literature investigates partitioning and transmutation
in a European context, with \glspl{ADS} and Gen-IV reactors \cite{fazio_study_2013,
alvarez-velarde_analysis_2008},
to reduce radiotoxicity for disposal.
However, little recent work considers synergistic international spent fuel arrangements.
This work finds that a collaborative strategy can reduce the
need to construct additional \glspl{LWR} in France, if 
the \glspl{SFR} are as commercially competitive as recent work
suggests they may be \cite{zhao_improving_2009}.